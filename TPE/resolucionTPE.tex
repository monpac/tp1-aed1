\documentclass[spanish,a4paper]{article} 

\usepackage{babel}
\usepackage{soul}
\usepackage[latin1]{inputenc}
%\usepackage{bbm}
\usepackage{framed}
\input{../macros/Algo1Macros}

\begin{document}

\materia{Algoritmos y Estructura de Datos I}
\cuatrimestre{1}
\anio{2015}

\fecha{1 de Abril de 2015}

\nombre{\LARGE TPE - Flores vs Vampiros}

\titulotp


\section{Tipos}

\input{tipos/tipos.tex}

\section{Flor}
\input{tipos/flor.tex}

\begin{problema}{nuevaF}{v : \ent, cP : \ent, hs : [Habilidad]}{Flor}
\requiere{sinRepetidos(hs)}
\requiere{v == 100 \textsf{div} (\longitud{hs} + 1) \land cP == \IfThenElse{\en{Atacar, hs}}{12 \textbf{div} \longitud{hs}}{0}}
\asegura{v == vida(res)}
\asegura{cP == cuantoPega(res)}
\asegura{mismos(hs, habilidades(res))}
\end{problema}

\begin{problema}{vidaF}{f: Flor}{\ent}
\asegura{res == vida(f)}
\end{problema}

\begin{problema}{cuantoPegaF}{f: Flor}{\ent}
\asegura{res == cuantoPega(f)}
\end{problema}

\begin{problema}{habilidadesF}{f: Flor}{[Habilidad]}
\asegura{mismos(res, habilidades(f))}
\end{problema}



\section{Vampiro}
\input{tipos/vampiro.tex}

\begin{problema}{nuevoV}{cv : ClaseVampiro, v : \ent, cP : \ent}{Vampiro}
\requiere[vidaEnRango]{v \geq 0 \land v \leq 100}
\requiere[pegaEnSerio]{cP > 0}
\asegura{clase(res) == cV}
\asegura{vida(res) == v}
\asegura{cuantoPega(res) == cP}
\end{problema}

\begin{problema}{claseVampiroV}{v : Vampiro}{ClaseVampiro}
\asegura{res == clase(v)}
\end{problema}

\begin{problema}{vidaV}{v : Vampiro}{\ent}
\asegura{res == vida(v)}
\end{problema}

\begin{problema}{cuantoPegaV}{v : Vampiro}{\ent}
\asegura{res == cuantoPega(v)}
\end{problema}

% problemas igual a Flor

\section{Nivel}
\begin{tipo}{Nivel}{}
	\observador{ancho}{n: Nivel}{\ent}
	\observador{alto}{n: Nivel}{\ent}
	\observador{turno}{n: Nivel}{\ent}
	\observador{soles}{n: Nivel}{\ent}
	\observador{flores}{n: Nivel}{[(Flor, Posicion, Vida)]}
	\observador{vampiros}{n: Nivel}{[(Vampiro, Posicion, Vida)]}
	\observador{spawning}{n: Nivel}{[(Vampiro, \ent, \ent)]}
	\invariante[valoresRazonables]{ancho(n) > 0 \land alto(n) > 0 \land soles(n) \geq 0 \land turno(n) \geq 0}
	\invariante[posicionesValidas]{(\forall f \selec flores(n))(1 \leq prm(sgd(f)) \leq ancho(n) \land 1 \leq sgd(sgd(f)) \leq alto(n)) \land (\forall v \selec vampiros(n))(1 \leq prm(sgd(v)) \leq ancho(n) \land 1 \leq sgd(sgd(v)) \leq alto(n))}
	\invariante[spawningOrdenado]{(\forall i, j \selec \rangoca{0}{\longitud{spawning(n)}}, i < j) peso(spawning(n)[i], n) < peso(spawning(n)[j], n)}
	\invariante[necesitoMiEspacio]{(\forall i,j \selec [0..|flores(n)|), i \neq j) sgd(flores(n)_i) \neq sgd(flores(n)_j)}
	\invariante[vivosPeroNoTanto]{vidaFloresOk(flores(n)) \land vidaVampirosOk(vampiros(n))}
	\invariante[spawneanBien]{(\forall t \selec spawning(n))sgd(t)\geq 1 \land sgd(t) \leq alto(n) \land trd(t)\geq 0}
\end{tipo}


\begin{problema}{nuevoN}{an : \ent, al : \ent, s : \ent, spaw : [(Vampiro, \ent, \ent)]}{Nivel}
\end{problema}

\begin{problema}{anchoN}{n : Nivel}{\ent}
\asegura{res == ancho(n)}
\end{problema}

\begin{problema}{altoN}{n : Nivel}{\ent}
\asegura{res == alto(n)}
\end{problema}


\begin{problema}{turnoN}{n : Nivel}{\ent}
\asegura{res == turno(n)}
\end{problema}

\begin{problema}{solesN}{n : Nivel}{\ent}
\asegura{res == soles(n)}
\end{problema}

\begin{problema}{floresN}{n : Nivel}{[(Flor, Posicion, Vida)]}
\asegura{mismos(res, flores(n))}
\end{problema}

\begin{problema}{vampirosN}{n : Nivel}{[(Vampiro, Posicion, Vida)]}
\end{problema}

\begin{problema}{spawningN}{n : Nivel}{[(Vampiro, \ent, \ent)]}
\end{problema}

\begin{problema}{comprarSoles}{n: Nivel, s : \ent}{}
\end{problema}

\begin{problema}{obsesivoCompusilvo}{n: Nivel}{\bool}
\end{problema}

\begin{problema}{agregarFlor}{n: Nivel, f : Flor, p : Posicion}{}
\end{problema}

\noindent\aux{terminado}{n: Nivel}{\bool}{}

\begin{problema}{pasarTurno}{n: Nivel}{}
\end{problema}




\section{Juego}
\input{tipos/juego.tex}

\begin{problema}{floresJ}{j: Juego}{[Flor]}
\asegura{mismos(result, flores(j))}
\end{problema}

\begin{problema}{vampirosJ}{j: Juego}{[Vampiro]}
\asegura{mismos(result, vampiros(j))}
\end{problema}

\begin{problema}{nivelesJ}{j: Juego}{[Nivel]}
\asegura{mismos(result, niveles(j))}
\end{problema}

\begin{problema}{agregarNivelJ}{j: Juego, n: Nivel, i: \ent}{}
\modifica{j}
\requiere{0 \leq i \land i \leq \longitud{niveles(j)}}
\requiere[nivelValido]{turno(n) == 0 \land \longitud{flores(n)} == 0 \land \longitud{vampiros(n)} == 0}
\asegura{indice(niveles(j), i) == n}
\asegura{niveles(j) \rangoca{0}{i} \masmas niveles(j) \rangoaa{i}{\longitud{niveles(j)}} == niveles(pre(j))}
\end{problema}

\begin{problema}{estosSalenFacil}{j: Juego}{[Nivel]}
\end{problema}

\begin{problema}{jugarNivel}{j: Juego, n: Nivel, i: \ent}{}
\end{problema}

\begin{problema}{altoCheat}{j: Juego, i: \ent}{}
\end{problema}

\begin{problema}{muyDeExactas}{j: Juego}{\bool}
\end{problema}

\section{Auxiliares}
\aux{vidaFloresOk}{fs: [(Flor, Posicion, Vida)]}{\bool}{(\forall f \selec fs) trd(f) > 0 \land trd(f) \leq vida(prm(f))}
\aux{vidaVampirosOk}{fs: [(Vampiro, Posicion, Vida)]}{\bool}{(\forall f \selec fs) trd(f) > 0 \land trd(f) \leq vida(prm(f))}
\aux{floresIguales}{x, y}{\bool}{mismos(habilidades(x), habilidades(y))}
\end{document}
